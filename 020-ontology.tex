%! TEX root = **/000-main.tex
% vim: spell spelllang=en:

\section{Ontology}%
\label{sec:ontology}

\subsection{TBOX definition}%
\label{sub:tbox}

% B1. TBOX definition
% ===================
% In case you used any other tool, include a graphical representation in the pdf
% file (e.g., using https://gra.fo/) and include the rdf, rdfs or owl file
% generated by the tool in the zip file to upload and following the naming
% notation explained (IMPORTANT: the lecturer should not install any additional
% tool to validate this part).

\subsection{ABOX definition}%
\label{sub:abox}

% B2. ABOX definition
% ===================
% - In the main pdf file, explain the methodology used to define your ABOX from
%   non-semantic data.
% - In the zip file, include your code to create the ABOX programmatically
%   (e.g., Jena or RDFLib code) and the resulting ABOX file (rdf, rdfs or owl).
%   These files must follow the naming notation explained.

\subsection{Inference regime}%
\label{sub:inference}

% B3. Create the final ontology
% =============================
% Besides that, you need to provide two additional outputs for this section
% (both must be included in the main pdf file of the deliverable):
% - Specify the inference regime entailment you are considering. Briefly explain
%   what rdf:type links you saved to explicitly generate thanks to reasoning.
% - Provide a summary table of your instances. Compute simple statistics about
%   the resulting knowledge graph. For example, the number of classes, the
%   number of properties, number of instances for the main classes and number of
%   triples using the main properties.

\begin{table}[H]
\centering
\caption{Summary of instances}
\label{tab:summary}
\begin{tabular}{lr}
  \toprule
  Property & Value \\
  \midrule
  Number of classes & 10\\
  Number of properties & 10\\
  Number of instances of the main class & 10\\
  Number of triples using the main properties & 10\\
  \dots & 10\\
  \bottomrule
\end{tabular}
\end{table}

\subsection{Querying the ontology}%
\label{sub:querying}

% B4. Querying the ontology
% =========================
% In the main pdf file, provide the following SPARQL queries (explicitly state
% any assumption you make):
% - Find all Authors.
% - Find all properties whose domain is Author.
% - Find all properties whose domain is either Conference or Journal.
% - Find all the papers written by a given author that where published in
%   database conferences.

\begin{minted}{sparql}
SELECT ?s ?p ?o
WHERE {
  ?s ?p ?o .
}
\end{minted}

